\documentclass[letterpaper,11pt]{article}

\oddsidemargin 0.0in
\evensidemargin 0.0in
\textwidth 6.5in
%\headheight 0.0in

\usepackage{graphics}
\usepackage{amsmath}
\usepackage{indentfirst}
\usepackage{tabularx}
\usepackage{graphicx}
\usepackage{url}
\usepackage{appendix}
\usepackage{verbatim}
\usepackage{lscape}
\usepackage{rotating}
\usepackage{longtable}



\DeclareMathOperator{\var}{var}
\DeclareMathOperator{\cov}{cov}

\usepackage{Sweave}
\begin{document}

\title{\texttt{multilevelPSA}: An R Package for Multilevel Propensity Score Analysis}
\author{Jason M. Bryer\\
\small{}jbryer@bryer.org}
\date{\today}

\maketitle

\abstract{The use of propensity score analysis (Rosenbaum \& Rubin, 1983) has gained increasing popularity for the estimation of causal effects within observational studies. However, its use in situations where data is multilevel, or clustered, is limited (Arpino \& Mealli, 2008). This talk will introduce the multilevelPSA (Bryer, 2011) package for R that provides functions for estimating propensity scores for large datasets using logistic regression and conditional inference trees. Furthermore, a set of graphical functions that extends the framework of visualizing propensity score analysis introduced by Helmreich and Pruzek (2009) to multilevel analysis will be discussed. An application for estimating the effects of private schools on reading, mathematics, and science outcomes from the Programme for International Student Assessment (PISA; Organization for Economic Co-operation and Development, 2009) is provided.
\ \\ \ \\
\noindent Keywords: \textit{PSA, propensity score analysis, multilevel, graphics}}



\section{Introduction}


\begin{Schunk}
\begin{Sinput}
> install.packages('multilevelPSA', repos=c('http://R-Forge.R-project.org', 
+ 'http://cran.r-project.org'), dep=TRUE)
\end{Sinput}
\end{Schunk}

\begin{Schunk}
\begin{Sinput}
> library(multilevelPSA)
\end{Sinput}
\end{Schunk}


\section{Programme for International Student Achievement}

\url{http://www.pisa.oecd.org/}

\begin{Schunk}
\begin{Sinput}
> download.file('http://multilevelpsa.r-forge.r-project.org/pisa/pisa.student.Rdata', 
+ 'pisa.student.Rdata')
> download.file('http://multilevelpsa.r-forge.r-project.org/pisa/pisa.school.Rdata',
+ 'pisa.school.Rdata')
\end{Sinput}
\end{Schunk}


\begin{Schunk}
\begin{Sinput}
> load('../../../../pisa.student.Rdata')
> load('../../../../pisa.school.Rdata')
> school = school.orig[,c('COUNTRY', "CNT", "SCHOOLID",
+ 	"SC02Q01", #Public (1) or private (2)
+ 	"STRATIO" #Student-teacher ration    
+ )]
> names(school) = c('COUNTRY', 'CNT', 'SCHOOLID', 'PUBPRIV', 'STRATIO')
> school$SCHOOLID = as.integer(school$SCHOOLID)
\end{Sinput}
\end{Schunk}

% latex table generated in R 2.14.0 by xtable 1.6-0 package
% Fri Nov 11 12:52:50 2011
\begin{longtable}{rrr}
\caption{Number of Private and Public Schools by Country} \\ 
  \hline
  Public & Private & Missing \\ \endfirsthead \multicolumn{3}{l}{{...continued from previous page}}\\ \hline Public & Private & Missing  \\ \hline \endhead \hline
164 &  17 &   0 \\ 
  158 &   4 &   0 \\ 
  141 &  58 &   0 \\ 
  217 & 136 &   0 \\ 
  234 &  39 &   9 \\ 
   89 & 189 &   0 \\ 
  812 &  98 &  37 \\ 
  174 &   4 &   0 \\ 
  896 &  76 &   6 \\ 
   80 & 103 &  17 \\ 
  137 &  15 &   0 \\ 
   97 &  61 &   0 \\ 
  222 &  51 &   2 \\ 
  154 &   4 &   0 \\ 
  234 &  15 &  12 \\ 
  231 &  50 &   4 \\ 
  168 &   7 &   0 \\ 
  191 &  12 &   0 \\ 
    0 &   0 & 168 \\ 
  202 &  11 &  13 \\ 
  173 &  11 &   0 \\ 
   10 & 140 &   1 \\ 
  163 &  24 &   0 \\ 
  112 &   2 &  17 \\ 
   85 &  98 &   0 \\ 
   57 &  87 &   0 \\ 
  140 &  31 &   5 \\ 
  987 &  84 &  26 \\ 
  135 &  51 &   0 \\ 
  191 &   8 &   0 \\ 
  181 &  29 &   0 \\ 
   99 &  58 &   0 \\ 
  167 &   6 &   0 \\ 
  181 &   3 &   0 \\ 
   10 &   2 &   0 \\ 
  193 &   2 &   1 \\ 
   30 &   9 &   0 \\ 
    3 &  42 &   0 \\ 
  1332 & 200 &   3 \\ 
   50 &   2 &   0 \\ 
   69 & 113 &   4 \\ 
  153 &  10 &   0 \\ 
  186 &   4 &   7 \\ 
  136 &  40 &  12 \\ 
  189 &  51 &   0 \\ 
  166 &  19 &   0 \\ 
  185 &  29 &   0 \\ 
   88 &  59 &   6 \\ 
  158 &   1 &   0 \\ 
  212 &   1 &   0 \\ 
  184 &   3 &   3 \\ 
  167 &   4 &   0 \\ 
  172 &  17 &   0 \\ 
  336 &   5 &   0 \\ 
  512 & 359 &  18 \\ 
  159 &  30 &   0 \\ 
  399 &  22 &   5 \\ 
  204 &  26 &   0 \\ 
  120 &  32 &   6 \\ 
   44 & 146 &   0 \\ 
  148 &  17 &   0 \\ 
  169 &   1 &   0 \\ 
  439 &  17 &  26 \\ 
  154 &  11 &   0 \\ 
  193 &  39 &   0 \\ 
   \hline
\hline
\label{ppxtab}
\end{longtable}
% latex table generated in R 2.14.0 by xtable 1.6-0 package
% Fri Nov 11 12:52:50 2011
\begin{longtable}{lll}
\caption{Covariates Used for Propensity Score Estimations} \\ 
  \hline
  Variable & Name & Description \\ \endfirsthead \multicolumn{3}{l}{{...continued from previous page}}\\ \hline Variable & Name & Description  \\ \hline \endhead \hline
CNT & CNT & Country \\ 
  SCHOOLID & SchoolId & SchoolID \\ 
  StIDStd & StudentId & Student ID \\ 
  ST01Q01 & Grade & Grade \\ 
  ST04Q01 & Sex & Sex \\ 
  ST05Q01 & Attend & Attend \\ 
  ST06Q01 & Age & Age \\ 
  ST07Q01 & Repeat & Repeat \\ 
  ST08Q01 & Mother & At home mother \\ 
  ST08Q02 & Father & At home father \\ 
  ST08Q03 & Brother & At home brothers \\ 
  ST08Q04 & Sister & At home sisters \\ 
  ST08Q05 & GrandPa & At home grandparents \\ 
  ST08Q06 & Other & At home others \\ 
  ST10Q01 & MomEd & Mother highest schooling \\ 
  ST12Q01 & MomJob & Mother current job status \\ 
  ST14Q01 & DadEd & Father highest schooling \\ 
  ST16Q01 & DadJob & Father current job status \\ 
  ST19Q01 & Lang & Language at home \\ 
  ST20Q01 & Desk & Desk \\ 
  ST20Q02 & OwnRoom & Own room \\ 
  ST20Q03 & StudyPl & Study place \\ 
  ST20Q04 & Computer & Computer \\ 
  ST20Q05 & Software & Software \\ 
  ST20Q06 & Internet & Internet \\ 
  ST20Q07 & Lit & Literature \\ 
  ST20Q08 & Poetry & Poetry \\ 
  ST20Q09 & Art & Art \\ 
  ST20Q10 & TxtBooks & Textbooks \\ 
  ST20Q12 & Dict & Dictionary \\ 
  ST20Q13 & DishW & Dishwasher \\ 
  ST20Q14 & DVD & DVD \\ 
  ST21Q01 & CellPh & How many cellphones \\ 
  ST21Q02 & TVs & How many TVs \\ 
  ST21Q03 & nComp & How many computers \\ 
  ST21Q04 & nCars & How many cars \\ 
  ST21Q05 & nBaths & How many rooms bath or shower \\ 
  ST22Q01 & nBooks & How many books \\ 
  ST23Q01 & Reading & Reading enjoyment time \\ 
  ST31Q01 & EnrichLang & Enrich in test language \\ 
  ST31Q02 & EnrichMath & Enrich in mathematics \\ 
  ST31Q03 & EnrichScie & Enrich in science \\ 
  ST31Q05 & RemedialLang & Remedial in test language \\ 
  ST31Q06 & RemedialMath & Remedial in mathematics \\ 
  ST31Q07 & RemedialScie & Remedial in science \\ 
  ST32Q01 & LangLessons & Out of school lessons in test language \\ 
  ST32Q02 & MathLessons & Out of school lessons maths \\ 
  ST32Q03 & ScieLessons & Out of school lessons in science \\ 
   \hline
\hline
\label{covariates}
\end{longtable}

\end{document}
